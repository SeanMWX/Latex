\subsection{Some ML examples in practice}
\begin{enumerate}
    \item Autonomous cars
    \item The Robosail project
    \item The Robot Scientist
    \item Infra Watch, "Hoolandse brug" - the bridge
    \item Language learning
    \item Automating manual tasks
\end{enumerate}

\subsection{Machine Learning}
\textbf{Definition} of machine learning: it is the study of how to make programs improve their performance on certain tasks from own (experience).
\\In this case:
\begin{itemize}
    \item "performance" = speed, accuracy
    \item "experience" = earlier observations
\end{itemize}
\textbf{Machine Learning vs. other AI} \\
In \textbf{machine learning}, the key is \textbf{data}, examples of questions and their answer; observations of earlier attempts to solve the problem
\\ In \textbf{inductive inference}, it is reasonsing from \textbf{specific} to \textbf{general}, statistics: sample -> population; from \textbf{concrete observations} -> \textbf{general theory}

\subsection{Machine Learning learning landscape}
\begin{outline}
    \1 tasks
        \2 clustering
        \2 classification
        \2 regression
        \2 reinforcement learning
    \1 techniques
        \2 Convex optimization
        \2 Matrix factorization
        \2 Transfer learning
        \2 Learning theory
        \2 Greedy search
    \1 models
        \2 automata
        \2 neural network
        \2 deep learning
        \2 statistical relational learning
        \2 decision trees
        \2 support vector machines
        \2 nearest neighbors
        \2 rule learners
        \2 bayesian learning
        \2 probabilisitc graphical models
    \1 applications
        \2 natural language processing
        \2 vision
        \2 speech
    \1 related courses
        \2 neural computing
        \2 support vector machine
        \2 uncertainty in AI
        \2 data mining
        \2 genetic algorithms and evolutionary computing
\end{outline}

\subsection{Some basic concepts and terminology}
\begin{outline}
    \1 Predictive learning
        \2 Definition: learn a model that can predict a particular property/ attribute/ variable from inputs
        \2 Binary classification: distinguish instances of class C from other instances
        \2 Classification: assign a class C (from a given set of classes) to an instances
        \2 Regression: assign a numerical value to an instance
        \2 multi-label classification: assign a set of labels (from a given set) to an instance
        \2 multivariate regression: assign a vector of numbers to an instances
        \2 multi-target prediction: assign a vector of values (numerical, categorical) to an instances
    \1 Descriptive learning
        \2 Definition: given a dataset, describe certain patterns in the dataset, or in the population it is drawn from
    \1 Typical tasks in ML
        \2 function learning: learn a function X->Y taht fits the given data
        \2 distribution learning: distribution learning
            \3 parametric: the function family of the distribution is known, we only need to estimate its parameters
            \3 non-parametric: no specific function family assumed
            \3 generative: generate new instances by random sampleing from it
            \3 discriminative: conditional probability distribution
    \1 Explainable AI (XAI)
        \2 Definition: means that the decisions of an AI system can be explained
        \2 Two different levels:
            \3 We understand the (learned) model
            \3 We understand the individual decision
\end{outline}

\subsection{Input formats (predictive learning)}
\begin{outline}
    \1 Set
        \2 training set: a set of examples, instance descriptions that include the target property (a.k.a. labeled instances)
        \2 prediction set: a set of instance descriptions that do not include the target property ('unlabeled' instances)
        \2 prediction task: predict the label of the unlabeled instances
    \1 Outcome of learning process
        \2 transductive learning: the predictions themselves
        \2 inductive learning: a function that can predict the label of any unlabeled instance
    \1 Explainable AI
        \2 interpretable: can be intepred
        \2 black-box: non-interpretable
    \1 Learning
        \2 Supervised learning: from labeled
        \2 Unsupervised learning: from unlabeled
        \2 Semi-supervised learning: from a few labeled and many unlabeled
    \1 Format of input data
        \2 input is often assumed to be a set of instances that are all described using the same variables (features, attributes)
        \2 i.i.d.: independent and identically distributed
            \3 tabular data (NN)
            \3 sequences
            \3 trees
            \3 graph
            \3 raw data: learning meaningful feaures from raw data
            \3 knowledge: inductive logic programming
\end{outline}

\subsection{Output formats, methods (predictive learning)}
The \textbf{output} of a learning system is a model.
\begin{outline}
    \1 output
        \2 parametrized functions
        \2 ocnjunctive concepts: a conjuntive concept is expressed as a set of conditions, all of which must be true
        \2 rule sets (if...then...else...)
        \2 decision trees
        \2 neural networks
        \2 probabilisitc graphical models
    \1 search methods
        \2 discrete spaces - methods: hill-climbing, best-first
        \2 continuous spaces - methods: gradient descent
    \1 typically
        \2 model structure not fixed in advanced - discrete
        \2 fixed model structure, tune numerical parameters - continuous
    \1 hypothesis space
        \2 definition: all possible instances
        \2 for robot example: \{B,R,M,?\} x \{S,T,?\} x \{L,W,?\} x \{1,2,?\}
    \1 Verson space
        \2 using candidate elimination
        \2 pros
            \3 can be used for discrete hypothesis spaces
            \3 search for all solutions, rather than just one, in an efficient manner
            \3 importance of generality ordering
        \2 cons
            \3 not robust to noise
            \3 only conjunctive concepts
\end{outline}


